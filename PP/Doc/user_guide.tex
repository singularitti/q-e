\documentclass[12pt,a4paper]{article}
\def\version{7.3}
\def\qe{{\sc Quantum ESPRESSO}}

\usepackage{html}

% BEWARE: don't revert from graphicx for epsfig, because latex2html
% doesn't handle epsfig commands !!!
\usepackage{graphicx}

\textwidth = 17cm
\textheight = 24cm
\topmargin =-1 cm
\oddsidemargin = 0 cm

\def\pwx{\texttt{pw.x}}
\def\cpx{\texttt{cp.x}}
\def\configure{\texttt{configure}}
\def\PWscf{\texttt{PWscf}}
\def\CP{\texttt{CP}}
\def\PHonon{\texttt{PHonon}}
\def\PostProc{\texttt{PostProc}}
\def\make{\texttt{make}}

\begin{document} 
\author{}
\date{}

\def\qeImage{../../Doc/quantum_espresso}

\title{
  \includegraphics[width=5cm]{\qeImage} \\
  % title
  \Huge \PostProc\ User's Guide (v.\version) 
}

\maketitle

\tableofcontents

\section{Introduction}

This guide covers the usage of \PostProc, version \version: 
an open-source package for postprocessing of data produced by
\PWscf\ and \CP.

This guide assumes that you know the physics 
that \PostProc\ describes and the methods it implements.
It also assumes  that you have already installed,
or know how to install, \qe. If not, please read
the general User's Guide for \qe, found in 
subdirectory \texttt{Doc/} of the main \qe\ directory,
or consult the web site:\\
\texttt{http://www.quantum-espresso.org}.

Further documentation, beyond what is provided 
in this guide, can be found in the directory
\texttt{PP/Doc/}, containing a copy of this guide.
People who want to contribute to \qe\ should read the
Wiki pages on GitLab: \texttt{https://gitlab.com/QEF/q-e/-/wikis}.

\section{People and terms of use}

The \PostProc\ package was originally developed by Stefano Baroni, 
Stefano de Gironcoli, Andrea Dal Corso (SISSA), Paolo Giannozzi 
(Univ. Udine), and many others. We mention in particular: 
\begin{itemize}
\item Dong Yang and Qin Liu (JSG) for calculation of DORI (10.1021/ct500490b)
      and for spin-polarized ELF;
\item Minsu Ghim (Seoul National U.) for Ji Hoon Ryoo's spin-current matrix
      elements (Phys. Rev. B 99, 235113) for spin Hall conductivity using
      Wannier interpolation, in pw2wannier.f90;
\item Yang Jiao, Elsebeth Schr\"oder, Per Hyldgaard (Chalmers) for the
  \texttt{ppacf.x} code;
\item Alberto Otero-de-la-Roza for the \texttt{pw2critic.x} utility;
\item Junfeng Qiao for improvements to \texttt{plotband.x};
\item Olivia Pulci, Adriano Mosca Conte, Davide Grassano (RomaII)
      for the \texttt{pw2gw} utility;
\item Cyrille Barreteau and Alexander Smogunov (CEA) for 
      magnetic anisotropy with the Force Theorem in \texttt{projwfc.f90};
\item Andrea Benassi (SISSA) for the \texttt{epsilon} utility,
      Tae Yun Kim and Cheol-Hwan Park (Seoul National University)
      for fixes to it;
\item Dmitry Korotin (Inst. Met. Phys. Ekaterinburg) for the
      \texttt{wannier\_ham} utility;
\item Georgy Samsonidze (Bosch Research) for the interface
      with the Berkeley GW code, Fangzhou Zhao (Berkeley) 
      for its extension to hybrid and meta-GGA functionals;
\item The late Prof. Eyvaz Isaev for the Fermi Surface code;
\item Natalie Holzwarth (WFU) for the PAW projection in code
     \texttt{projwfc.f90};
\item Takashi Koretsune and Florian Thoele (ETHZ) for noncollinear 
      magnetisation support with USPP and PAW pseudopotentials in 
      code \texttt{pw2wannier.f90}.
\item Leopold Talirz (U.York) for extensions and fixes to \texttt{pp.x}.
\end{itemize}

\PostProc\ is free software, released under the 
GNU General Public License. See:\\
\texttt{http://www.gnu.org/licenses/old-licenses/gpl-2.0.txt}, 
or the file License in the distribution).
    
%%%%%%%%%%%%%%%%%%%%%%%%%%%%%%%%%%%%%%%%%%%%%%%%%%%%%%%%%%%%%%%%%
\input ../../Doc/quote.tex
%%%%%%%%%%%%%%%%%%%%%%%%%%%%%%%%%%%%%%%%%%%%%%%%%%%%%%%%%%%%%%%%%

\section{Compilation}

\PostProc \ is part of the \qe \ distribution
and depends upon \PWscf\ for compilation.
For instruction on how to download and compile \qe, please refer 
to the general Users' Guide, available in file \texttt{Doc/user\_guide.pdf}
under the main \qe\ directory, or in web site 
\texttt{http://www.quantum-espresso.org}.

Once \qe\ is correctly configured, \PostProc\ can be compiled by
just typing \texttt{make pp}, from the main \qe\ directory;
or typing \make\ from the \texttt{PP/} subdirectory.
Several executable codes are produced in \texttt{PP/bin}
and linked to \texttt{bin/}.

\section{Usage}

All codes for which input documentation is not explicitly mentioned below
have some documentation in the header of the fortran sources.
In the following, subdirectories containing examples are found in
\texttt{PP/examples/}; ``Example N'' stands for subdirectory 
\texttt{PP/examples/exampleN/}.

All quantities whose dimensions are not explicitly specified are in
RYDBERG ATOMIC UNITS. Charge is "number" charge (i.e. not multiplied 
by $e$); potentials are in energy units (i.e. they are multiplied by 
$e$).

\subsection{Plotting selected quantities}
  
The main postprocessing code \texttt{pp.x} extracts the specified data
from the data files produced by \PWscf\ (\pwx\ executable) or \CP\ 
(\cpx\ executable); prepares data for plotting by writing them into 
formats that can be read by several plotting programs.

Quantities that can be read or calculated are:
\begin{quote}
      charge density\\
      spin polarization\\
      various potentials\\
      local density of states at $E_F$\\
      local density of electronic entropy\\
      STM images\\
      selected squared wavefunction\\
      ELF (electron localization function)\\
      RDG (reduced density gradient)\\
      integrated local density of states
\end{quote}
Various types of plotting (along a line, on a plane, three-dimensional, polar)
and output formats (including the popular cube format) can be specified.
Moreover data can be saved to an intermediate (formatted) file so that
more data sets can be summed or subracted in a later run.
The output files can be directly read by the free plotting system Gnuplot
(1D or 2D plots), or by code \texttt{plotrho.x} that comes with \PostProc\ 
and produces PostScript 2D plots,
or by advanced plotting software XCrySDen (3D plots).

See file \texttt{PP/Doc/INPUT\_PP.*} for a detailed description of the input
for code \texttt{pp.x}.
See Example 01 for an example of a charge density plot, Example 03
for an example of STM image simulation.

\paragraph{Planar averages}
Code \texttt{plan\_avg.x} calculates planar averages of Kohn-Sham orbitals.
Input documentation is in the header of\texttt{PP/src/plan\_avg.f90}.

Code \texttt{average.x} calculates planar averages of quantities produced
by \texttt{pp.x} (e.g. potentials, charge, magnetization densities).
Note that \texttt{average.x} reads the intermediate file produced
by \texttt{pp.x}, not data files produced by \pwx. Examples of usage 
of \texttt{average.x} can be found in \texttt{PP/examples/WorkFct\_example/} 
and in \texttt{PP/examples/dipole\_example/}.

\paragraph{All-electron charge}
\texttt{pawplot.x} produces plots of the all-electron charge
for PAW calculations. Input documentation in the header of
\texttt{PP/src/pawplot.f90}. 

\subsection{About Bader's analysis}
In \texttt{http://theory.cm.utexas.edu/henkelman/code/bader/} 
one can find a software that performs Bader's analysis starting 
from charge on a regular grid. One should use PAW to compute the
charge density. The required "cube" format can be produced using 
\texttt{pp.x} (info by G. Lapenna who has successfully used this 
technique, but adds: ``Problems occur with polar X-H bonds or in
all cases where the zero-flux of density comes too close to atoms 
described with pseudo-potentials"). This code should perform 
decomposition into Voronoi polyhedra as well, in place of obsolete
code  \texttt{voronoy.x} (removed from distribution since v.4.2).
Alternatively, you can use \textsl{CRITIC2}, available at
\texttt{https://github.com/aoterodelaroza/critic2}, which can
read directly \texttt{pw.x} output and ``XSF'' files. \textsl{CRITIC2}
functionaly include Bader's AIM, ELF, laplacian of density and
potentials, non-covalente interaction (NCI) plots and much more.

\subsection{Band structure, Fermi surface}

The code \texttt{bands.x} reads data file(s), extracts eigenvalues,
regroups them into bands (the algorithm used to order bands and to resolve
crossings may not work in all circumstances, though). The output is written
to a file in a simple format that can be directly read and converted to
plottable format by auxiliary code
\texttt{plotband.x}. Unpredictable plots may results if k-points are not 
in sequence along lines, or if two consecutive points are the same. 
The code \texttt{bands.x} performs as well a 
symmetry analysis of the band structure. For a complete input description,
see \texttt{Doc/INPUT\_bands.*}. See Example 01, Example 04 and Example 06 
for simple band plots.

The plotting of Fermi surfaces can be performed using code \texttt{fs.x}.
The resulting file in .bxsf format can be read and plotted
using XCrySDen. See Example 02 for an example of Fermi surface 
visualization (Ni, including the spin-polarized case).

\subsection{Projection over atomic states, DOS, projected band structure}

The code \texttt{projwfc.x} calculates projections of wavefunctions
over atomic orbitals. The atomic wavefunctions are those contained
in the pseudopotential file(s). The L\"owdin population analysis (similar to
Mulliken analysis) is presently implemented. The projected DOS (or PDOS:
the DOS projected onto atomic orbitals) can also be calculated and written
to file(s). More details on the input data are found in file
\texttt{PP/Doc/INPUT\_PROJWFC.*}. The ordering of the various 
angular momentum components (defined in routine \texttt{ylmr2.f90})
is as follows:
$P_{0,0}(t)$, $P_{1,0}(t)$, $P_{1,1}(t)cos\phi$, $P_{1,1}(t)sin\phi$,
 $P_{2,0}(t)$, $P_{2,1}(t)cos\phi$, $P_{2,1}(t)sin\phi$, 
 $P_{2,2}(t)cos2\phi$, $P_{2,2}(t)sin2\phi$
and so on, where $P_{l,m}$=Legendre Polynomials, 
$t = cos\theta = z/r$, $\phi= atan(y /x)$.

Data produced by code \texttt{projwfc.x} can be further 
analysed using auxiliary codes \texttt{sumpdos.x} (sums selected PDOS
by specifying the names of files containing the desired PDOS: type 
\texttt{sumpdos.x -h} or look into the source code for more details) 
and \texttt{plotproj.x} . A more sophisticated tools is the script
\texttt{PP/tools/sum\_states.py}, by Julen Larrucea: documentation in
\texttt{http://larrucea.eu/sum\_states-py-2/}.

The total electronic DOS can also be calculated by code \texttt{dos.x},
whose complete input documentation is in \texttt{PP/Doc/INPUT\_DOS.*}
See Example 02 for total and projected electronic DOS calculations,
-and for projected band structure;
see Example 03 for projected and local DOS calculations.

The DOS projected over {\em molecular} states (e.g. for a molecule on
a surface system) can be computed using code \texttt{molecularpdos.x}
(courtesy of Guido Fratesi). See file \texttt{PP/Doc/INPUT\_MOLDOS.*}
for input documentation and directory \texttt{PP/examples/MolDos\_example/} for
an example.

The calculation of magnetic anisotropy using the Force Theorem is described
in the following paper:
https://journals.aps.org/prb/abstract/10.1103/PhysRevB.90.205409. An
example and a README can be found in \texttt{PP/examples/ForceTheorem\_example/}

\subsection{Color plot of the Fermi velocity and the orbital character
  on Fermi surfaces}

You can plot any quantity on Fermi surfaces as a color plot 
by using \verb|fermisurfer| program\footnote{http://osdn.jp/projects/fermisurfer/}.
\verb|fermi_velocity.x| and \verb|fermi_proj.x| are used 
to generate an input file for \verb|fermisurfer| from the output
of \pwx or \verb|projwfc.x|.

\verb|fermi_velocity.x| generates a color-plot of Fermi velocity.
You use it as follows:
\begin{enumerate}
\item Run \pwx with \verb|K_POINT automatic|.
\item Run 
\begin{verbatim}
$ fermi_velocity.x -in {pw.x input file}
\end{verbatim}  
\item \verb|vfermi.frmsf| is generated
\end{enumerate}

\verb|fermi_proj.x| generates a color plot of an orbital character.
You use it as follows:
\begin{enumerate}
\item Run \pwx with \verb|K_POINT automatic|.
\item Run \verb|projwfc.x| just to generate \verb|{prefix}.save/atomic_proj.*|.
\item Run 
\begin{verbatim}
$ fermi_proj.x -in {input file}
\end{verbatim}
Input-file format is as follows:
\begin{verbatim}
&PROJWFC
 {The same as the input of projwfc.x}
/
{Number of target wavefunctions}
{Index of target WFC1} {Index of target WFC2} {Index of target WFC3} ...
\end{verbatim}
It generates 
$\sum_{i=1}^{n_{\rm target}} |\langle \varphi_{{\rm target}(i)}^{\rm atom}|\varphi_{n k} \rangle|^2$, 
where $n_s$ and ${\rm target}(i)$ are
the number of the target wavefunctions
and the indices of target wavefunctions, respectively.
\item The above quantity is written into \verb|"proj.frmsf"|,
  which can be read by FermiSurfer program.
\end{enumerate}

There is an example of \verb|fermi_velocity.x| and \verb|fermi_proj.x|
in \verb|fermisurf_example/|.

\subsection{Wannier functions}

There are several Wannier-related utilities in \PostProc:
\begin{enumerate}
\item The "Poor Man Wannier" code \texttt{pmw.x}, to be used
in conjunction with DFT+U calculations: see Example 05.
\item The interface with Wannier90 code, \texttt{pw2wannier.x}:
see the documentation in \texttt{W90/} (you may install the 
Wannier90 plug-in via \texttt{make w90} ). For spin-current
matrix elements, implemented in routine \texttt{compute\_shc}:
``it writes .sIu and .sHu files used for
WANNIER-BERRI (https://github.com/stepan-tsirkin/wannier-berri/),
and also will be utilized through postw90.x
(https://github.com/manxkim/wannier90/tree/SHC/src) in Wannier90.
In WANNIER-BERRI, .sHu and .sIu files can be used to calculate the
quantity "opt\_SHCryoo". In Wannier90, add "berry\_task = shc" and
"shc\_ryoo=.true.". in the input parameters of postw90.x. They
activate the calculation of SHC using .sHu and .sIu.''
\item The \texttt{wannier\_ham.x} code generates a model Hamiltonian 
in Wannier functions basis: see \texttt{PP/examples/WannierHam\_example/}.
\end{enumerate}
Note that the \texttt{wfdd.x} code has been moved to \CP.

\subsection{Interfaces to/from other code}

Codes \texttt{pw2bgw.x} convert data files from \pwx\ to a format suitable
for usage by the Berkeley GW code. See file \texttt{Doc/INPUT\_pw2bgw.*}
for input data documentation. Code \texttt{bgw2pw.x}, performing the
inverse conversion, no longer works: a copy that worked for the old
file format is kept for reference in \texttt{bgw2pw.f90.orig}.

Code \texttt{pw2gw.x} converts data files from \pwx\ to a format suitable 
for usage by another GW code, computes optical properties in single-particle 
approach (Fermi Golden Rule). See file \texttt{Doc/INPUT\_pw2gw.html}
for input data documentation, directory \texttt{pw2gw\_example/}
for an example of usage.

Code \texttt{open\_grid.x} writes Kohn-Sham orbitals for the complete
k-point grid (not symmetry-independent points only) in real space.
Useful for further processing. It can be used to generate the
Kohn-Sham state data required in \texttt{pw2wannier.x} and Wannier90
from the initial SCF calculation, bypassing the non-SCF calculation
step.

Code \texttt{pw2critic.x} is an interface to the CRITIC2 code by
Alberto Otero-de-la-Roza. This program creates a \texttt{pwc} file
containing the Kohn-Sham orbitals from an SCF calculation (or from the
output of \texttt{open\_grid.x}). These orbitals are used for
post-processing in CRITIC2.

Code \texttt{pw\_export.f90} no longer works and is no longer present.

\subsection{Other tools}

\paragraph{Exchange-correlation}
Code \texttt{ppacf.x} computes the coupling constant dependency of the
exchange correlation potential $E_{xc,\lambda}, \lambda \in [0:1]$
and the spatial distribution of the exchange-correlation energy density
and kinetic correlation energy density according to:
Y. Jiao, E. Schr\"oder, and P. Hyldgaard, Phys. Rev. B 97, 085115 (2018).
See \texttt{PP/Doc/INPUT\_PPACF.html}.

\paragraph{Wavefunction conversion}
Code \texttt{wfck2r.x} converts Kohn-Sham orbitals from reciprocal to real 
space. It is a useful starting point if you need to access wavefunctions
and perform postprocessing operations that are not implemented in \qe.

\paragraph{Dielectric function}
Code \texttt{epsilon.x} calculates RPA frequency-dependent complex dielectric 
function. Documentation is in file \texttt{Doc/eps\_man.tex}.

\paragraph{Core-level shifts}
Code \texttt{initial\_state.x} calculates the initial state contribution
to the Core-level shift. See \texttt{CLS\_IS\_example/} for
an example, and \texttt{CLS\_FS\_example/} for the corresponding
final state calculation of Core-level shifts.

\section{Troubleshooting}

Almost all problems in \qe\ arise from incorrect input data and result in
error stops. Error messages should be self-explanatory, but unfortunately
this is not always true. If the code issues a warning messages and continues,
pay attention to it but do not assume that something is necessarily wrong in
your calculation: most warning messages signal harmless problems.

\paragraph{Some postprocessing codes complain that they do not find some files}
Most likely you are not reading the correct data files, or you are not
following the correct procedure for postprocessing.

For Linux PC clusters in parallel execution: in at least some versions
of MPICH, the current directory is set to the directory where the executable
code resides, instead of being set to the directory where the code is executed.
This MPICH weirdness may cause unexpected failures in some postprocessing
codes that expect a data file in the current directory. Workaround: use
symbolic links, or copy the executable to the current directory.

\end{document}
